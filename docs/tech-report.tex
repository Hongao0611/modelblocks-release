\documentclass[12pt]{report}
\usepackage{hyperref}
\setcounter{secnumdepth}{4}
%\setcounter{tocdepth}{1}
\usepackage{amsmath}
\usepackage{times}
\usepackage{helvet}
\usepackage{txfonts}  %% WS: NEEDED TO KEEP SOME VERBATIM LINES FROM EXCEEDING PAGE WIDTH
\usepackage{natbib}
%%\usepackage{epsfig}
%%\usepackage{psfig}
%\usepackage{array}
%\usepackage{needspace}

\usepackage{vmargin}
\setpapersize{USletter}
\setmarginsrb{1.0in}{1.0in}{1.0in}{0.6in}{0pt}{0pt}{0pt}{0.4in}

% HOW TO USE THE ABOVE:
%\setmarginsrb{leftmargin}{topmargin}{rightmargin}{bottommargin}{headheight}{headsep}{footheight}{footskip}
%\raggedbottom
% paragraphs indent & skip:
%\parindent  0.3cm
%\parskip    -0.01cm

%\usepackage{amsfonts}
\usepackage{tikz}
\usetikzlibrary{fit,shapes, backgrounds}
\usepackage[T1,safe]{tipa}

% hyphenation:
\sloppy

% notes-style paragraph spacing and indentation:
\usepackage{parskip}
\setlength{\parindent}{0cm}

%\bibliographystyle{apacite}
\bibliographystyle{apalike}

%\usepackage{../../styles/schuler}
\newcommand{\tuple}[1]{\langle #1 \rangle}

\usepackage{enumitem}
\setlist[itemize]{leftmargin=1em}

\usepackage{titlesec}
\usepackage{etoolbox}

\makeatletter
\patchcmd{\ttlh@hang}{\parindent\z@}{\parindent\z@\leavevmode}{}{}
\patchcmd{\ttlh@hang}{\noindent}{}{}{}
\makeatother

\titlespacing\section{0pt}{0pt}{0pt}
\titlespacing\subsection{0pt}{0pt}{0pt}
\titlespacing\subsubsection{0pt}{0pt}{0pt}

\titleformat{\chapter}[display]
  {\normalfont\sffamily\Huge\bfseries}
  {\chaptertitlename\ \thechapter}{20pt}{\Huge}
\titleformat{\section}
  {\normalfont\sffamily\Large\bfseries}
  {\thesection}{1em}{}
\titleformat{\subsection}
  {\normalfont\sffamily\large\bfseries}
  {\thesubsection}{1em}{}
\titleformat{\subsubsection}
  {\normalfont\sffamily\bfseries}
  {\thesubsubsection}{1em}{}

%%%%%%%%%%%%%%%%%%%%%%%%%%%%%%%%%%%%%%%%%%%%%%%%%%%%%%%%%%%%%%%%%%%%%%%%%%%%%%%%

\def\blue{\color{blue}}
\def\magenta{\color{magenta}}
\def\red{\color{red}}
\def\dec{-}
\def\inc{+}
%\def\dec{\text{-}}

%%%%%%%%%%%%%%%%%%%%%%%%%%%%%%%%%%%%%%%%%%%%%%%%%%%%%%%%%%%%%%%%%%%%%%%%%%%%%%%%

\begin{document}

\begin{titlepage}
      \begin{tikzpicture}
      \begin{pgfinterruptboundingbox}
      \begin{pgfonlayer}{background}
      \fill[cyan] (-2,-1) rectangle (0,-3);
      \fill[cyan!80] (0,-1) rectangle (2,-3);
      \fill[cyan!60] (-2,-3) rectangle (0,-5);
      \end{pgfonlayer}
      \end{pgfinterruptboundingbox}
      \end{tikzpicture}
      \centering\sffamily
      {\scshape\large The Ohio State University \par}
      \vspace{3cm}
      {\Huge Modelblocks\par}
      \vspace{2cm}
      {\large\itshape Repository documentation\par}
      \vfill
      {\normalsize Last updated: \today\par}
\end{titlepage}

\newcommand{\dtype}[1]{{\blue\tuple{\text{\em #1}}}}
\newcommand{\dexpr}[1]{{\blue\tuple{\text{\tt #1}}}}
\newcommand{\dcode}[1]{{\blue\text{`{\tt #1}'}}}
\newcommand{\dvar}[1]{{\blue\tuple{#1}}}

\bigskip

\tableofcontents




%%%%%%%%%%%%%%%%%%%%%%%%%%%%%%%%%%%%%%%%%%%%%%%%%%%%%%%%%%%%%%%%%%%%%%%%%%%%%%%%
%%%%%%%%%%%%%%%%%%%%%%%%%%%%%%%%%%%%%%%%%%%%%%%%%%%%%%%%%%%%%%%%%%%%%%%%%%%%%%%%
%%%%%%%%%%%%%%%%%%%%%%%%%%%%%%%%%%%%%%%%%%%%%%%%%%%%%%%%%%%%%%%%%%%%%%%%%%%%%%%%
\chapter{Overview}
%%%%%%%%%%%%%%%%%%%%%%%%%%%%%%%%%%%%%%%%%%%%%%%%%%%%%%%%%%%%%%%%%%%%%%%%%%%%%%%%
%%%%%%%%%%%%%%%%%%%%%%%%%%%%%%%%%%%%%%%%%%%%%%%%%%%%%%%%%%%%%%%%%%%%%%%%%%%%%%%%
%%%%%%%%%%%%%%%%%%%%%%%%%%%%%%%%%%%%%%%%%%%%%%%%%%%%%%%%%%%%%%%%%%%%%%%%%%%%%%%%

Modelblocks is a collection of recipes related to cognitive modeling of sentence processing.

It is implemented as a set of directories containing Makefiles, consisting of:
%
\begin{itemize}

\item a {\bf config} directory (`{\blue\tt config}') which contains a set of `{\blue\tt user-*-directory.txt}' files

      (these files contain locations of external code or data not part of Modelblocks);

\item a set of {\bf resource} directories, prefixed with `{\blue\tt resource-}'

      (recipes and code in these directories are designed to be reusable); and

\item a set of {\bf project} directories (those not prefixed with `{\blue\tt resource-}', other than `{\blue\tt config}')

      (recipes and code in these directories are designed to perform a single experiment).

\end{itemize}

Each resource or project directory contains a `{\blue\tt Makefile}' file, containing its relevant recipes.

Resource and project directories also contain scripts, code, and curated data used in their recipes.

\bigskip

Makefiles in project directories usually begin with a default item, which replicates the main result of the project, followed by a colon, e.g.:
%
{\magenta\begin{verbatim}
#### default item: the main product of this directory...
genmodel/wsj02to21.gcg15.linetrees:
\end{verbatim}
}
%
This allows the main result of the project to be replicated by moving to that directory and typing:
%
{\blue\begin{verbatim}
make
\end{verbatim}
}

Project Makefiles then contain a preamble, consisting of some useful make commands:
%
{\magenta\begin{verbatim}
#### preamble...
.SUFFIXES:
.SECONDEXPANSION:
\end{verbatim}
}
%
followed by a list of defined subdirectories which included files will make use of:
%
{\magenta\begin{verbatim}
#### directories used by included items...
BIN := bin
GENMODEL := genmodel
\end{verbatim}
}

Project Makefiles can then inherit recipes from resource directories:
%
{\magenta\begin{verbatim}
#### included make items...
include $(dir $(CURDIR))resource-general/Makefile
\end{verbatim}
}

Resource directories {\red cannot} include other resource directories because of limitations in the GNU make {\blue\tt include} command.

Please note that because Modelblocks primarily uses GNU Make to automate the code in experiments, dependencies to targets are created before the targets themselves. Thus, even though the recipes below are presented in API format with inputs and outputs, \textbf{you do not need to make the inputs before making an output target}. It should be sufficient to just type

\vspace{1em}

{\blue\tt make <stem>.<extension>}

\vspace{1em}

from the relevant directory to produce an output --- make will handle the rest. For that reason, generally the most important information in the following entries is the ``Stem template'' field, which explains how to write and delimit parameters to targets so that they can be successfully made.

Generally, targets with extensions like `widgetgadgets' will be formatted as plain-text matrices with rows for `widgets' (delimited by newlines) and columns for `gadgets' (delimited by spaces).

E.g.\ `tokdecs' files are formatted as matrices with a row for each `tok(en)' (delimited by newlines) and a column for each `dec(ision)' (delimited by spaces).






%%%%%%%%%%%%%%%%%%%%%%%%%%%%%%%%%%%%%%%%%%%%%%%%%%%%%%%%%%%%%%%%%%%%%%%%%%%%%%%%
%%%%%%%%%%%%%%%%%%%%%%%%%%%%%%%%%%%%%%%%%%%%%%%%%%%%%%%%%%%%%%%%%%%%%%%%%%%%%%%%
%%%%%%%%%%%%%%%%%%%%%%%%%%%%%%%%%%%%%%%%%%%%%%%%%%%%%%%%%%%%%%%%%%%%%%%%%%%%%%%%
\chapter{Repository Policy}
%%%%%%%%%%%%%%%%%%%%%%%%%%%%%%%%%%%%%%%%%%%%%%%%%%%%%%%%%%%%%%%%%%%%%%%%%%%%%%%%
%%%%%%%%%%%%%%%%%%%%%%%%%%%%%%%%%%%%%%%%%%%%%%%%%%%%%%%%%%%%%%%%%%%%%%%%%%%%%%%%
%%%%%%%%%%%%%%%%%%%%%%%%%%%%%%%%%%%%%%%%%%%%%%%%%%%%%%%%%%%%%%%%%%%%%%%%%%%%%%%%

Modelblocks has a local OSU git repository, which {\blue\tt ling.osu.edu} account holders can access with:\!\!\!\!\!\!\!
%
{\blue\begin{verbatim}
git clone ssh://<username>@ling.osu.edu/home/compling/modelblocks-repository
\end{verbatim}
}

Here are a few guidelines about repository etiquette within this project:
%
\begin{enumerate}

\item {\red VERY IMPORTANT: Do not push any code or data that is available elsewhere, especially if the license is not open-source!
      If you do, someone will have to purge the comp ling copy of the repo and ask all users to delete their local repos!}

      This should instead be handled using `{\blue\tt config/user-*-directory.txt}' pointer files.

\item Only add project or resource directories to the repo that are likely to be useful to other people.

      Course projects, etc., can be maintained in private repos within your modelblocks directory.

\item Only create resource directories for recipes that are shared by more than one project.

\end{enumerate}




%%%%%%%%%%%%%%%%%%%%%%%%%%%%%%%%%%%%%%%%%%%%%%%%%%%%%%%%%%%%%%%%%%%%%%%%%%%%%%%%
%%%%%%%%%%%%%%%%%%%%%%%%%%%%%%%%%%%%%%%%%%%%%%%%%%%%%%%%%%%%%%%%%%%%%%%%%%%%%%%%
%%%%%%%%%%%%%%%%%%%%%%%%%%%%%%%%%%%%%%%%%%%%%%%%%%%%%%%%%%%%%%%%%%%%%%%%%%%%%%%%
\chapter{Terms}
%%%%%%%%%%%%%%%%%%%%%%%%%%%%%%%%%%%%%%%%%%%%%%%%%%%%%%%%%%%%%%%%%%%%%%%%%%%%%%%%
%%%%%%%%%%%%%%%%%%%%%%%%%%%%%%%%%%%%%%%%%%%%%%%%%%%%%%%%%%%%%%%%%%%%%%%%%%%%%%%%
%%%%%%%%%%%%%%%%%%%%%%%%%%%%%%%%%%%%%%%%%%%%%%%%%%%%%%%%%%%%%%%%%%%%%%%%%%%%%%%%

Modelblocks assumes the following definitions of terms:
%
\begin{itemize}

\item `{\blue line}': a sentence, title, or caption, which is syntactically independent of its neighbors.

      (e.g.\ `{\magenta\tt April in Poetry}')

\item `{\blue token}' (`{\blue tok}'): an (arbitrary) elementary unit of syntactic annotation (may not be a word).

      (e.g.\ `{\magenta\tt they}', `{\magenta\tt do}', `{\magenta\tt n't}', `{\magenta\tt ?}')

\item `{\blue item}': a word as delimited in a source corpus.\footnote{In some cases, parser input may need be tokenized more finely than the experimental stimuli are (e.g.\ splitting out punctuation and contractions). In these cases, the `token'/`item' distinction becomes important. Tokens are always nested inside items.}

      (e.g.\ `{\magenta\tt they}', `{\magenta\tt don't?}')

%\item `prediction' (`pred'): an elementary unit of happening in a model

%\item `outcome' (`out'): an element of happening in a model

%\item `observation' (`obs'): an observed event in human subject data

\item %`state': an individual step in processing a corpus.
      `{\blue decision}' (`{\blue dec}'): an elementary unit of (processing) activity, as defined in annotated corpora.\!\!

      (e.g.\ {\magenta a decision to fork at token 3})

\item `{\blue parameter}' (`{\blue par}'): the predictors and result of a decision according to a model.

      (e.g.\ `{\magenta\tt F 2 VP : 1}')

%\item `{\blue weight}': the value assigned to a parameter by a model.
%
%      (e.g.\ `{\magenta\tt 0.0049}')

\item `{\blue event}' (`{\blue ev}'): an elementary unit of stimulus in human subject data (i.e.\ regression targets).
      %`{\blue trial}': a unit of stimulus in human subject data (i.e.\ regression targets).

      (e.g.\ {\magenta exposure of subject 3 to region 22 of the stimulus})

\end{itemize}


%%%%%%%%%%%%%%%%%%%%%%%%%%%%%%%%%%%%%%%%%%%%%%%%%%%%%%%%%%%%%%%%%%%%%%%%%%%%%%%%
%%%%%%%%%%%%%%%%%%%%%%%%%%%%%%%%%%%%%%%%%%%%%%%%%%%%%%%%%%%%%%%%%%%%%%%%%%%%%%%%
%%%%%%%%%%%%%%%%%%%%%%%%%%%%%%%%%%%%%%%%%%%%%%%%%%%%%%%%%%%%%%%%%%%%%%%%%%%%%%%%
\chapter{RESOURCE-LINETREES}
%%%% FROM NOW ON, DON'T SHOW ITEM BULLETS
\renewcommand{\labelitemi}{}
\renewcommand{\labelitemii}{}
\renewcommand{\labelitemiii}{}
\renewcommand{\labelitemiv}{}
%%%%%%%%%%%%%%%%%%%%%%%%%%%%%%%%%%%%%%%%%%%%%%%%%%%%%%%%%%%%%%%%%%%%%%%%%%%%%%%%
%%%%%%%%%%%%%%%%%%%%%%%%%%%%%%%%%%%%%%%%%%%%%%%%%%%%%%%%%%%%%%%%%%%%%%%%%%%%%%%%
%%%%%%%%%%%%%%%%%%%%%%%%%%%%%%%%%%%%%%%%%%%%%%%%%%%%%%%%%%%%%%%%%%%%%%%%%%%%%%%%

{\tt RESOURCE-LINETREES} contains methods for generating and manipulating syntactic annotations and evaluating parser accuracy.

\section{\blue\tt \%.linetoks}

Extracts word sequences from an input {\blue\tt \%.linetrees} file (i.e.\ deletes structure from a phrase-structure tree).

\begin{itemize}
\item\textbf{File format:}
\begin{itemize}
\item\textit{Specification:}

\begin{itemize}
\item $\dtype{row} \rightarrow \Big( \ \dtype{token} \ \Big( \ \dcode{\ } \ \dtype{token} \ \Big)^* \ \Big)^?$
\end{itemize}

\item\textbf{Example:}

{\magenta\begin{verbatim}
The cat did n't sleep .
\end{verbatim}
}
\end{itemize}
\end{itemize}

\subsection{Recipe: {\blue\tt \%.linetoks}}

Extracts word sequences from input trees.

\begin{itemize}
      \item \textbf{Stem template:}\\
      $\dtype{corpus-specification} \ \dcode{.linetoks}$
      \item \textbf{Input(s):}
      \begin{itemize}
            \item $\dtype{corpus-specification} \ \dcode{.linetrees}$
      \end{itemize}
\end{itemize}

\subsection{Recipe: {\blue\tt \%.morph.linetoks}}

Uses Morfessor to segment input {\blue\tt \%.linetoks}.

\begin{itemize}
      \item \textbf{Stem template:}\\
      $\dtype{corpus-specification} \ \dcode{.morph.linetoks}$
      \item \textbf{Input(s):}
      \begin{itemize}
            \item $\dtype{corpus-specification} \ \dcode{.linetoks}$
      \end{itemize}
\end{itemize}

\subsection{Recipe: {\blue\tt \%.rev.linetoks}}

Reverses input {\blue\tt \%.linetoks}.

\begin{itemize}
      \item \textbf{Stem template:} $\dtype{corpus-specification} \ \dcode{.rev.linetoks}$
      \item \textbf{Input(s):}
      \begin{itemize}
            \item $\dtype{corpus-specification} \ \dcode{.linetoks}$
      \end{itemize}
\end{itemize}


\section{\blue\tt \%.linetrees}

Defines trees for each `line' (sentence, title, caption, etc.) in a corpus using a generalization of the Penn Treebank standard.

\begin{itemize}
\item\textbf{File format:}
\begin{itemize}
\item\textit{Specification:}
\begin{itemize}
\item $\dtype{row} \rightarrow \dtype{tree}$
\item $\dtype{tree}  \rightarrow \dcode{(} \ \dtype{node-label} \ \ \smash{\Big( \ \dcode{\ } \ \dtype{tree} \ \Big)^+} \ \dcode{)}$
\item $\dtype{tree}  \rightarrow \dtype{node-label}$
%\item $\dtype{node-label}  \rightarrow \dexpr{[a-z]*}$
\end{itemize}

\item\textit{Example:}
{\magenta\begin{verbatim}
(S (S (NP The cat) (VP did n't sleep)) (. .))
\end{verbatim}
}
\end{itemize}
\end{itemize}

\subsection{Recipe: {\blue\tt \%.linetrees}}


Generates a specification-conformant linetrees file from another file containing syntactic annotations.

\begin{itemize}
      \item \textit{Option 1: Converting external annotations}
      \begin{itemize}
            \item \textbf{Stem template:}\\
            $\dtype{corpus-specification} \ \dcode{.linetrees}$
            \item \textbf{Input(s):}
            \begin{itemize}
                  \item External annotation (handlers provided by associated {\tt RESOURCE-*} directory).
            \end{itemize}
      \end{itemize}
      \item \textit{Option 2: Converting from {\blue\tt \%.tokdecs}}
      \begin{itemize}
            \item \textbf{Stem template:}\\
            $\dtype{corpus-specification} \ \dcode{.linetrees}$
            \item \textbf{Input(s):}
            \begin{itemize}
                  \item $\dtype{corpus-specification} \ \dcode{.tokdecs}$
            \end{itemize}
      \end{itemize}
      \item \textit{Option 3: Parsing source text using a probability model}
      \begin{itemize}
            \item \textbf{Stem template:}\\
            $\dtype{test-set-specification} \ \dcode{.} \ \dtype{dash-delimited-trained-model-name} \ \dcode{-parsed.linetrees}$
            \item \textbf{Input(s):}
            \begin{itemize}
                  \item $\dtype{test-set-specification} \ \dcode{.linetoks}$
                  \item $\dtype{dot-delimited-trained-model-name} \ \dcode{.parweights}$
            \end{itemize}
      \end{itemize}
\end{itemize}

\subsection{Recipe: {\blue\tt \%.first.linetrees}}

Extracts the first $\dtype{number}$ of lines from the input.

\begin{itemize}
      \item \textbf{Stem template:}\\
      $\dtype{corpus-specification} \ \dcode{.} \ \dtype{number} \ \dcode{first.linetrees}$
      \item \textbf{Input(s):}
      \begin{itemize}
            \item $\dtype{corpus-specification} \ \dcode{.linetrees}$
      \end{itemize}
\end{itemize}

\subsection{Recipe: {\blue\tt \%.last.linetrees}}

Extracts the last $\dtype{number}$ of lines from the input.

\begin{itemize}
      \item \textbf{Stem template:}\\
      $\dtype{corpus-specification} \ \dcode{.} \ \dtype{number} \ \dcode{last.linetrees}$
      \item \textbf{Input(s):}
      \begin{itemize}
            \item $\dtype{corpus-specification} \ \dcode{.linetrees}$
      \end{itemize}
\end{itemize}

\subsection{Recipe: {\blue\tt \%.onward.linetrees}}

Extracts lines $\dtype{number}$ and onward from the input.

\begin{itemize}
      \item \textbf{Stem template:}\\
      $\dtype{corpus-specification} \ \dcode{.} \ \dtype{number} \ \dcode{onward.linetrees}$
      \item \textbf{Input(s):}
      \begin{itemize}
            \item $\dtype{corpus-specification} \ \dcode{.linetrees}$
      \end{itemize}
\end{itemize}

\subsection{Recipe: {\blue\tt \%.maxwords.linetrees}}

Extracts only lines from the input containing at most $\dtype{number}$ words.

\begin{itemize}
      \item \textbf{Stem template:}\\
      $\dtype{corpus-specification} \ \dcode{.} \ \dtype{number} \ \dcode{maxwords.linetrees}$
      \item \textbf{Input(s):}
      \begin{itemize}
            \item $\dtype{corpus-specification} \ \dcode{.linetrees}$
      \end{itemize}
\end{itemize}

\subsection{Recipe: {\blue\tt \%.nolabel.linetrees}}

Removes syntactic labels from the input.

\begin{itemize}
      \item \textbf{Stem template:}\\
      $\dtype{corpus-specification} \ \dcode{.nolabel.linetrees}$
      \item \textbf{Input(s):}
      \begin{itemize}
            \item $\dtype{corpus-specification} \ \dcode{.linetrees}$
      \end{itemize}
\end{itemize}

\subsection{Recipe: {\blue\tt \%.nopunc.linetrees}}

Removes punctuation from the input.

\begin{itemize}
      \item \textbf{Stem template:}\\
      $\dtype{corpus-specification} \ \dcode{.nopunc.linetrees}$
      \item \textbf{Input(s):}
      \begin{itemize}
            \item $\dtype{corpus-specification} \ \dcode{.linetrees}$
      \end{itemize}
\end{itemize}

\subsection{Recipe: {\blue\tt \%.fromdeps.linetrees}}

Converts dependency representations into constituency tree representations using the \cite{collinsetal99} algorith (produces the flattest constituency trees permitted by the dependency graph).

\begin{itemize}
      \item \textbf{Stem template:}\\
      $\dtype{corpus-specification} \ \dcode{.fromdeps.linetrees}$
      \item \textbf{Input(s):}
      \begin{itemize}
            \item $\dtype{corpus-specification} \ \dcode{.tokdeps}$
      \end{itemize}
\end{itemize}


\section{\blue\tt \%.lineitems}

Extracts word sequences from a corpus of psycholinguistic stimuli (where tokenizations generally differ from those used for parsing), preserving original tokenization.

\begin{itemize}
\item\textbf{File format:}
\begin{itemize}
\item\textit{Specification:}

\begin{itemize}
\item $\dtype{row} \rightarrow \Big( \ \dtype{item} \ \Big( \ \dcode{\ } \ \dtype{item} \ \Big)^* \ \Big)^?$
\end{itemize}

\item\textbf{Example:}

{\magenta\begin{verbatim}
The cat didn't sleep.
\end{verbatim}
}
\end{itemize}
\end{itemize}

\subsection{Recipe: {\blue\tt \%.lineitems}}

\begin{itemize}
      \item \textbf{Stem template:}\\
      $\dtype{corpus-specification} \ \dcode{.lineitems}$
      \item \textbf{Input(s):}
      \begin{itemize}
            \item External corpus
      \end{itemize}
\end{itemize}


\section{\blue\tt \%.conll}

Generates CoNLL-style dependencies from input {\blue\tt \%.linetrees} by assigning the most frequently co-occurring child as the head.

\begin{itemize}
\item\textbf{File format:} \textit{See CoNLL documentation}
\end{itemize}

\subsection{Recipe: {\blue\tt \%.conll}}

\begin{itemize}
      \item \textbf{Stem template:}\\
      $\dtype{corpus-specification} \ \dcode{.conll}$
      \item \textbf{Input(s):}
      \begin{itemize}
            \item $\dtype{corpus-specification} \ \dcode{.linetrees}$
      \end{itemize}
\end{itemize}

\section{\blue\tt \%.tokdeps}

Generates Stanford-style dependencies from input {\blue\tt \%.linetrees} by assigning the most frequently co-occurring child as the head.

\begin{itemize}
\item\textbf{File format:} \textit{See Stanford Dependencies documentation}
\end{itemize}

\subsection{Recipe: {\blue\tt \%.tokdeps}}

\begin{itemize}
      \item \textbf{Stem template:}\\
      $\dtype{corpus-specification} \ \dcode{.tokdeps}$
      \item \textbf{Input(s):}
      \begin{itemize}
            \item $\dtype{corpus-specification} \ \dcode{.linetrees}$
      \end{itemize}
\end{itemize}


\section{\blue\tt \%.syneval}

Generates human-readable report of differences between two {\blue\tt \%.linetrees} inputs (typically a gold and a hypothesized sequence of trees).

\begin{itemize}
\item\textbf{File format:} \textit{Human-readable}
\end{itemize}

\subsection{Recipe: {\blue\tt \%.syneval}}

\begin{itemize}
      \item \textbf{Stem template:}\\
      $\dtype{dash-delimited-test-set-name}$ $\dtype{dot-delimited-gold-edits}$ $\dcode{.}$ $\dtype{processing-specifications}$ $\dcode{.syneval}$
      \item \textbf{Input(s):}
      \begin{itemize}
            \item {\tt RESOURCE-GENERAL/srcmodel/new.prm}
            \item $\dtype{dot-delimited-test-set-name} \ \dtype{dot-delimited-gold-edits} \ \dcode{.linetrees}$ (control/baseline linetrees)
            \item  $\dtype{dot-delimited-test-set-name} \ \dcode{.} \ \dtype{processing-specifications} \ \dcode{.linetrees}$ (test linetrees)
      \end{itemize}
\end{itemize}

\section{\blue\tt \%.bootstrapsignif}

Generates human-readable report of statistical significance of difference between two {\blue\tt \%.syneval} files (typically a control and a test sequence of eval scores) obtained via bootstrap resampling.

\begin{itemize}
\item\textbf{File format:} \textit{Human-readable}
\end{itemize}

\subsection{Recipe: {\blue\tt \%.bootstrapsignif}}

\begin{itemize}
      \item \textbf{Stem template:}\\
      $\dtype{common-onset}$ $\dcode{..}$ $\dtype{distinct-ctrl}$ $\dcode{..}$ $\dtype{distinct-test}$ $\dcode{..}$ $\dtype{common-coda}$ $\dcode{.bootstrapsignif}$
      \item \textbf{Input(s):}
      \begin{itemize}
            \item {\tt RESOURCE-GENERAL/srcmodel/new.prm}
            \item  $\dtype{common-onset} \ \dcode{.} \ \dtype{distinct-ctrl} \ \dcode{.} \ \ \dtype{common-coda} \ \dcode{.syneval} \ \dtype{common-coda}$ (control/baseline syneval)
            \item  $\dtype{common-onset} \ \dcode{.} \ \dtype{distinct-test} \ \dcode{.} \ \dtype{common-coda} \ \dcode{.syneval} \ \dtype{common-coda}$ (test syneval)
      \end{itemize}
\end{itemize}


\section{\blue\tt \%(.d).constiteval.txt}

Alternative to {\blue\tt \%.syneval} with additional measures relevant to unsupervised PoS tagging, parsing, and word segmentation (e.g.\ tagging and consituent labeling V-Measures, segmentation accuracy, etc.). Can be used when gold and test tokenizations differ arbitrarily. Generates human-readable report of differences between two {\blue\tt \%.linetrees} inputs (typically a gold and a hypothesized sequence of trees).

\begin{itemize}
\item\textbf{File format:} \textit{Human-readable}
\end{itemize}

\subsection{Recipe: {\blue\tt \%(.d.)constiteval.txt}}

Without the {\blue\tt .d.} option, generates summary metrics for entire input corpus. With it, generates additional per-sentence metrics (more informative but consumes much more disk space).

\begin{itemize}
      \item \textbf{Stem template:}\\
      $\dtype{dash-delimited-test-set-name}$ $\dtype{dot-delimited-gold-edits}$ $\dcode{.}$ $\dtype{processing-specifications}$ $\dcode{.syneval}$
      \item \textbf{Input(s):}
      \begin{itemize}
            \item  $\dtype{distinct-ctrl} \ \dcode{.} \ \ \dtype{common-coda} \ \dtype{.} \ \dcode{.syneval} \ \dtype{common-coda}$ (control/baseline linetrees)
            \item  $\dtype{distinct-test} \ \dcode{.} \ \\dtype{common-coda} \ \dtype{.} \ dcode{.syneval} \ \dtype{common-coda}$ (test linetrees)
      \end{itemize}
\end{itemize}

\subsection{Recipe: {\blue\tt \%.constiteval.txt}}

\begin{itemize}
      \item \textbf{Stem template:}\\
      $\dtype{distinct-ctrl}$ $\dcode{..}$ $\dtype{distinct-test}$ $\dcode{..}$ $\dtype{common-coda}$ $\dcode{.constiteval.txt}$
      \item \textbf{Input(s):}
      \begin{itemize}
            \item  $\dtype{distinct-ctrl} \ \dcode{.} \ \dtype{common-coda}$ (control/baseline constiteval)
            \item  $\dtype{common-onset} \ \dcode{.} \ \dtype{distinct-test} \ \dcode{.} \ \dtype{common-coda}$ (test constiteval)
      \end{itemize}
\end{itemize}


\section{\blue\tt \%.constitevaltable.txt}

Generates human-readable summary table of multiple {\blue\tt \%.constiteval.txt} results that can be used for plotting.

\begin{itemize}
      \item \textbf{File format:}
      \begin{itemize}
            \item \textit{Specification}:
            \begin{itemize}
                  \item $\dtype{row} \rightarrow \Big( \ \dtype{index} \ \Big( \ \dcode{\ } \ \dtype{eval-measure} \ \Big)^* \ \Big)^?$
            \end{itemize}
      \end{itemize}
      \item \textbf{Example:}\\
      {\magenta\tt
      iter bracketingF1 segmentationF1 ...\\
      1 0.5611 0.9175 ...}
\end{itemize}

\subsection{Recipe: {\blue\tt \%.constitevaltable.txt}}

\begin{itemize}
      \item \textbf{Stem template:}\\
      $\dtype{corpus-specification}$ $\dcode{.}$ $\dtype{distinct-test}$ $\dcode{.}$ $\dtype{common-coda}$ $\dcode{.}$ $\dtype{constiteval-list}$ $\dcode{.constitevaltable.txt}$
      \item \textbf{Input(s):}
      \begin{itemize}
            \item  $\dtype{basename} \ \dcode{.constitevallist}$ (file containing paths to {\blue\tt \%.constiteval} files to aggregate, in most cases must be created by hand)
      \end{itemize}
\end{itemize}

\section{\blue\tt \%.learning\textunderscore curves}

Convenience target. Generates plots of a variety of learning curves from a {\blue\tt \%.constitevaltable.txt} file.

\begin{itemize}
      \item \textbf{File format:} \textit{Plot images}
\end{itemize}

\subsection{Recipe: {\blue\tt \%.learning\textunderscore curves}}

\begin{itemize}
      \item \textbf{Stem template:}\\
      $\dtype{constiteval-list-specification}$ $\dcode{.learning\textunderscore curves}$
      \item \textbf{Input(s):}
      \begin{itemize}
            \item $\dtype{constiteval-list-specification}$ $\dcode{.constitevallist.txt}$
      \end{itemize}
\end{itemize}



%%%%%%%%%%%%%%%%%%%%%%%%%%%%%%%%%%%%%%%%%%%%%%%%%%%%%%%%%%%%%%%%%%%%%%%%%%%%%%%%
%%%%%%%%%%%%%%%%%%%%%%%%%%%%%%%%%%%%%%%%%%%%%%%%%%%%%%%%%%%%%%%%%%%%%%%%%%%%%%%%
%%%%%%%%%%%%%%%%%%%%%%%%%%%%%%%%%%%%%%%%%%%%%%%%%%%%%%%%%%%%%%%%%%%%%%%%%%%%%%%%
\chapter{RESOURCE-GCG}
%%%%%%%%%%%%%%%%%%%%%%%%%%%%%%%%%%%%%%%%%%%%%%%%%%%%%%%%%%%%%%%%%%%%%%%%%%%%%%%%
%%%%%%%%%%%%%%%%%%%%%%%%%%%%%%%%%%%%%%%%%%%%%%%%%%%%%%%%%%%%%%%%%%%%%%%%%%%%%%%%
%%%%%%%%%%%%%%%%%%%%%%%%%%%%%%%%%%%%%%%%%%%%%%%%%%%%%%%%%%%%%%%%%%%%%%%%%%%%%%%%

{\tt RESOURCE-GCG} contains methods for generating and manipulating semantic annotations.

%%%%%%%%%%%%%%%%%%%%%%%%%%%%%%%%%%%%%%%%
\section{\blue\tt \%.gcg\#\#.linetrees}

GCG linetrees specify categorial predicate-argument markup for trees.
%
GCG linetrees conform to the following convention:

\begin{tabular}{l}
$\dtype{row}  \rightarrow \dtype{tree}$
\\
$\dtype{tree} \rightarrow \dtype{token}$
\\
$\dtype{tree} \rightarrow \dcode{(} \ \dtype{sign-type} \ \ {\Big( \ \dcode{\ } \ \dtype{tree} \ \Big)^+} \ \dcode{)}$
\end{tabular}

where sign types consist of a clause type and one or more dependencies:

\begin{tabular}{l}
%\item $\dtype{node-label}  \rightarrow \dexpr{[a-z]*}$
%\\
$\dtype{sign-type}  \rightarrow \dtype{clause-type} \ {\Big( \ \dtype{dependency} \ \Big)^*}$
\end{tabular}

and dependencies conform to one of the following formats:

\begin{tabular}{ll}
$\dtype{dependency} \rightarrow \dcode{-} \ \dexpr{[abcdghirv]} \ \dtype{sign-type}$                             & (simple dependent signs)
\\
$\dtype{dependency} \rightarrow \dcode{-} \ \dexpr{[abcdghirv]} \ \dcode{\{} \ \dtype{sign-type} \ \dcode{\}}$   & (complex dependent signs)
\\
$\dtype{dependency} \rightarrow \dcode{-l} \ \dexpr{[AMCN]}$                                                     & (operation labels)
\\
$\dtype{dependency} \rightarrow \dcode{-x}$                                                                      &  (indicates terminal sign)
\end{tabular}

For example:

{\magenta\begin{verbatim}
(S (S (N (N-b{N-aD} The) (N-aD cat)) (V-aN slept)) (. .))
\end{verbatim}
}

%%%%%%%%%%%%%%%%%%%%
\subsection{Recipe: {\blue\tt \%.gcg\#\#.linetrees}}

Making any GCG linetrees file that conforms to the naming convention:

$\dtype{corpus-specification} \ \dcode{.gcg} \ \dtype{version-year} \ \dcode{.linetrees}$

will generate and then use the following files:

$\dtype{corpus-specification} \ \dcode{.linetrees}$


%%%%%%%%%%%%%%%%%%%%%%%%%%%%%%%%%%%%%%%%
\section{\blue\tt \%.morphed.linetrees}
\label{sect:morph}

`Morphed' trees are trees with morphological markup.
%
This markup extends the dependency type of GCG trees as rewrite rules from inflected forms to base forms:

$\dtype{dependency} \rightarrow \dcode{-x} \ \underbrace{\dtype{clause-type} \ \dcode{\%:} \ \dtype{prefix} \ \Big( \ \dcode{\%} \ \dtype{affix} \ \Big)^n}_\text{inflected form}
                             \ \dcode{$|$} \ \underbrace{\dtype{clause-type} \ \dcode{\%:} \ \dtype{prefix} \ \Big( \ \dcode{\%} \ \dtype{affix} \ \Big)^n}_\text{base form}$

$\dtype{dependency} \rightarrow \dcode{-x} \ \underbrace{\dtype{clause-type} \ \dcode{\%} \ \dtype{suffix}}_\text{inflected form}
                             \ \dcode{$|$} \ \underbrace{\dtype{clause-type} \ \dcode{\%} \ \dtype{suffix}}_\text{base form}$


\textit{Example:}
{\magenta\begin{verbatim}
(S (S (N (N-b{N-aD} The) (N-aD-xN%s|N% cats)) (V-aN-xV%ept|B%eep slept)) (. .))
\end{verbatim}
}
%\end{itemize}
%\end{itemize}

%%%%%%%%%%%%%%%%%%%%
\subsection{Recipe: {\blue\tt \%.morphed.linetrees}}

%Annotate morphological operations.

Making any refconts that conforms to the naming convention:

$\dtype{corpus-specification} \ \dcode{.morphed.linetrees}$

will generate and then use the following files:

$\dtype{corpus-specification} \ \dcode{.linetrees}$


%%%%%%%%%%%%%%%%%%%%%%%%%%%%%%%%%%%%%%%%
\section{{\blue\tt \%.refconts} (formerly {\red\tt \%.varconts})}
\label{sect:refconts}

`Refconts' files list contexts for each referential state in a cued-association graph.
%
Refconts files conform to the following specification:

$\dtype{row} \rightarrow \dtype{referent-identifier} \ \Big( \ \dcode{\ } \ \dtype{context} \ \Big)^*$

$\dtype{context} \rightarrow \dtype{predicate} \ \dcode{\_} \ \Big( \ \dtype{argument-number} \ \Big( \ \dcode{-} \ \dtype{argument-number} \ \Big)^? \ \Big)^?$

Contexts list the forward (positive) and backward (negative) dependencies traversed from each nearby predicate to each referent in a cued-association graph.
%
For example:
%
{\magenta\begin{verbatim}
165-0201 BeingACar_1-1 Driving_
\end{verbatim}
}
%
indicates document 165, line 2, token 1 is a `Driving' predicate that has a first argument which is the first argument of a `BeingACar' predicate.

%%%%%%%%%%%%%%%%%%%%
\subsection{Recipe: {\blue\tt \%.refconts}}

Making any refconts that conforms to the naming convention:

$\dtype{corpus-specification} \ \dcode{.refconts}$

will generate and then use the following files:

$\dtype{corpus-specification} \ \dcode{.cg.linedeps}$


%%%%%%%%%%%%%%%%%%%%%%%%%%%%%%%%%%%%%%%%%%%%%%%%%%%%%%%%%%%%%%%%%%%%%%%%%%%%%%%%
%%%%%%%%%%%%%%%%%%%%%%%%%%%%%%%%%%%%%%%%%%%%%%%%%%%%%%%%%%%%%%%%%%%%%%%%%%%%%%%%
%%%%%%%%%%%%%%%%%%%%%%%%%%%%%%%%%%%%%%%%%%%%%%%%%%%%%%%%%%%%%%%%%%%%%%%%%%%%%%%%
\chapter{RESOURCE-INCRSEM}
%%%%%%%%%%%%%%%%%%%%%%%%%%%%%%%%%%%%%%%%%%%%%%%%%%%%%%%%%%%%%%%%%%%%%%%%%%%%%%%%
%%%%%%%%%%%%%%%%%%%%%%%%%%%%%%%%%%%%%%%%%%%%%%%%%%%%%%%%%%%%%%%%%%%%%%%%%%%%%%%%
%%%%%%%%%%%%%%%%%%%%%%%%%%%%%%%%%%%%%%%%%%%%%%%%%%%%%%%%%%%%%%%%%%%%%%%%%%%%%%%%

{\tt RESOURCE-INCRSEM} contains methods for incrementally processing semantic annotations.

%%%%%%%%%%%%%%%%%%%%%%%%%%%%%%%%%%%%%%%%
\section{{\blue\tt \%typed.linetrees}}

There can be many different predicate types in a cued-association structure, which can make statistics over predicates very sparse.
%
Predicates in cued-association structures derived from \citet{nguyenetal12} GCG trees can therefore automatically be organized into latent-variable hypernym clusters.
%
These hypernym clusters are annotated as morphological rules on preterminal categories of GCG trees, which, together with other morphological rules which transform $\dtype{preterminal-category} \ \dcode{:} \ \dtype{inflected-form}$ pairs into $\dtype{lemma-category} \ \dcode{:} \ \dtype{lemma-form}$ pairs as described in Section~\ref{sect:morph}, then transform these $\dtype{lemma-category} \ \dcode{:} \ \dtype{lemma-form}$ pairs into $\dtype{hypernym-category} \ \dcode{:y} \ \dtype{hypernym-number}$ pairs (or vice versa if applied in reverse).

%%%%%%%%%%%%%%%%%%%%
\subsection{Recipe: {\blue\tt \%typed.linetrees}}

Making any file that conforms to the naming convention:

$\dtype{corpus} \ \dcode{.} \ \dtype{\#-clusters} \ \dcode{typed.linetrees}$

will generate and then use the following files:

$\dtype{corpus} \ \dcode{.linetrees}$


%%%%%%%%%%%%%%%%%%%%%%%%%%%%%%%%%%%%%%%%
\section{{\blue\tt \%\_semprocmodel}}

%%%%%%%%%%%%%%%%%%%%
\subsection{Recipe: {\blue\tt \%\_semprocmodel}}

Making any file that conforms to the naming convention:

$\dtype{corpus} \ \dcode{.} \ \dtype{model-params} \ \dcode{\_} \ \dtype{\#-f-regularization} \ \dcode{\_} \ \dtype{\#-j-regularization} \ \dcode{\_semprocmodel}$

will generate and then use the following files:

$\dtype{corpus} \ \dcode{.} \ \dtype{model-params} \ \dcode{\_.} \ \dtype{\#-f-regularization} \ \dcode{\_100\_fmlrmodel}$

$\dtype{corpus} \ \dcode{.} \ \dtype{model-params} \ \dcode{\_.} \ \dtype{\#-j-regularization} \ \dcode{\_100\_jmlrmodel}$

$\dtype{corpus} \ \dcode{.} \ \dtype{model-params} \ \dcode{\_.pcptmodel}$

$\dtype{corpus} \ \dcode{.} \ \dtype{model-params} \ \dcode{\_.wcptmodel}$

$\dtype{corpus} \ \dcode{.} \ \dtype{model-params} \ \dcode{\_.acptmodel}$

$\dtype{corpus} \ \dcode{.} \ \dtype{model-params} \ \dcode{\_.bcptmodel}$




%%%%%%%%%%%%%%%%%%%%%%%%%%%%%%%%%%%%%%%%%
%\section{{\blue\tt \%\_ymodel}}
%
%Ymodels define clusters for contexts of referential states in cued-association graphs.
%%
%Ymodel files conform to the following specification:
%
%$\dtype{row} \rightarrow \dcode{Y\ } \ \dtype{referent-identifier} \ \dcode{\ :\ } \ \dtype{cluster-number} \ \dcode{\ =\ } \ \dtype{probability}$
%
%for priors over clusters at each referent, and
%
%$\dtype{row} \rightarrow \dcode{K\ } \ \dtype{cluster-number} \ \dcode{\ :\ } \ \dtype{context} \ \dcode{\ =\ } \ \dtype{probability}$
%
%%\noindent
%for likelihoods of contexts given clusters, where contexts are defined as in Section~\ref{sect:refconts} above.
%
%%%%%%%%%%%%%%%%%%%%%
%\subsection{Recipe: {\blue\tt \%\_ymodel}}
%
%Making any ymodel that conforms to the naming convention:
%
%$\dtype{corpus} \ \dcode{.} \ \dtype{\#-clusters} \ \dcode{\_} \ \dtype{\#-iterations} \ \dcode{\_} \ \dtype{alpha} \ \dcode{\_} \ \dtype{beta} \ \dcode{\_} \ \dtype{ref-start} \ \dcode{\_} \ \dtype{ref-end} \ \dcode{\_ymodel}$
%
%will generate and then use the following files:
%
%$\dtype{corpus} \ \dcode{.refconts}$
%
%applying a Latent Dirichlet Allocation with $\dtype{\#-clusters}$ clusters, $\dtype{\#-iterations}$ iterations, priors $\dtype{alpha}$ and $\dtype{beta}$ over clusters and contexts respectively, to referential states with contexts of the form `$\dtype{ref-start}$ \dots $\dtype{ref-end}$' in the file $\dtype{corpus}\text{\blue\tt .refconts}$.


%%%%%%%%%%%%%%%%%%%%%%%%%%%%%%%%%%%%%%%%%%%%%%%%%%%%%%%%%%%%%%%%%%%%%%%%%%%%%%%%
%%%%%%%%%%%%%%%%%%%%%%%%%%%%%%%%%%%%%%%%%%%%%%%%%%%%%%%%%%%%%%%%%%%%%%%%%%%%%%%%
%%%%%%%%%%%%%%%%%%%%%%%%%%%%%%%%%%%%%%%%%%%%%%%%%%%%%%%%%%%%%%%%%%%%%%%%%%%%%%%%
\chapter{RESOURCE-LCPARSE}
%%%%%%%%%%%%%%%%%%%%%%%%%%%%%%%%%%%%%%%%%%%%%%%%%%%%%%%%%%%%%%%%%%%%%%%%%%%%%%%%
%%%%%%%%%%%%%%%%%%%%%%%%%%%%%%%%%%%%%%%%%%%%%%%%%%%%%%%%%%%%%%%%%%%%%%%%%%%%%%%%
%%%%%%%%%%%%%%%%%%%%%%%%%%%%%%%%%%%%%%%%%%%%%%%%%%%%%%%%%%%%%%%%%%%%%%%%%%%%%%%%

{\tt RESOURCE-LCPARSE} contains methods for generating representations of incremental left-corner parsing operations.

\section{{\blue\tt \%.linedeps} (formerly {\red\tt \%.cuegraphs})}
Generates sorted set of dependencies for each `line' (sentence, title, caption, etc.) in a corpus. If present, `0' dependencies are used to indicate vertex types.

\begin{itemize}
      \item \textbf{File format:}
      \begin{itemize}
            \item \textit{Specification:}
            \begin{itemize}
                  \item {\small $\dtype{row}        \rightarrow \Big( \ \dtype{dependency} \ \Big( \ \dcode{\ } \ \dtype{dependency} \ \Big)^* \ \Big)^?$}
                  \item {\small $\dtype{dependency} \rightarrow \dtype{(source-)vertex} \ \dcode{,} \ \dtype{dependency-label} \ \dcode{,} \ \dtype{(destination-)vertex}$}
            \end{itemize}
            \item \textit{Example:}\\
            {\magenta\tt
            01,0,cats 02,0,sleep 02,1,01
            }
      \end{itemize}
\end{itemize}

TODO: Is this used? I don't see any recipes anywhere in MB with this extension.

\section{{\blue\tt \%.tokdecs}}

Generates list of modeled decisions for each token position in a corpus.

\begin{itemize}
      \item \textbf{File format:}
      \begin{itemize}
            \item \textit{Specification:}
            \begin{itemize}
                  \item {\small $\dtype{row} \rightarrow \dtype{word} \ \dcode{\ } \ \dtype{(preterminal-)sign} \ \dcode{\ } \ \dtype{fork} \ \dcode{\ } \ \dtype{join} \ \dcode{\ } \ \Big( \ \dtype{deriv-frag} \ \Big( \ \dcode{;} \ \dtype{deriv-frag} \ \Big)^* \ \Big)^?$}
                  %\item $\dtype{derivation-fragment-sequence} \rightarrow \Big( \ \dtype{derivation-fragment} \ \Big( \ \dcode{;} \ \dtype{derivation-fragment} \ \Big)^* \ \Big)^?$
                  \item {\small $\dtype{deriv-frag} \rightarrow \dtype{(top-)sign} \ \dcode{/} \ \dtype{(bottom-)sign}$
                  \item $\dtype{sign} \rightarrow \dcode{[} \ \Big( \ \dtype{referential-context} \ \Big( \ \dcode{,} \ \dtype{referential-context} \ \Big)^* \ \Big)^? \ \dcode{]:} \ \dtype{category-label}$}
                  %\item $\dtype{referential-context-set} \rightarrow \Big( \ \dtype{referential-context} \ \Big( \ \dcode{,} \ \dtype{referential-context} \ \Big)^* \ \Big)^?$
            \end{itemize}
            \item \textit{Example:}\\
            {\magenta\tt
            the [the\textunderscore 0]:D 1 0 [the\textunderscore 1]:NP/[the\textunderscore 1]:N\\
            very [very\textunderscore 0]:Adv 1 0 [the\textunderscore 1]:NP/[the\textunderscore 1]:N;[very\textunderscore 1]:AP/[very\textunderscore 1]:A\\
            }
      \end{itemize}
\end{itemize}

\subsection{Recipe: \blue\tt \%.fromlinetrees.tokdecs}

\begin{itemize}
      \item \textbf{Stem template:}\\
      $\dtype{corpus-specification} \ \dcode{.fromlinetrees.tokdecs}$
      \item \textbf{Input(s):}
      \begin{itemize}
            \item $\dtype{corpus-specification} \ \dcode{.linetrees}$
      \end{itemize}
\end{itemize}


\section{{\blue\tt \%.decpars}}

Generates list of parameterizations for training decisions. There are separate formats for ordinary conditional probability and logistic regression parameters.

\subsection{Recipe: \blue\tt \%.cpt.decpars}

Conditional probability decpars.

\begin{itemize}
      \item \textbf{File format:}
      \begin{itemize}
            \item \textit{Specification:}
            \begin{itemize}
                  \item {\small $\dtype{row} \rightarrow \dtype{decision-type} \ \ \smash{\Big( \ \dcode{\ } \ \dtype{predictor-value} \ \Big)^*} \ \dcode{ :} \ \ \smash{\Big( \ \dcode{\ } \ \dtype{result-value} \ \Big)^+}$}
            \end{itemize}
            \item \textit{Example:}\\
            {\magenta\tt
            P 2 0 1 VP : VT
            }
            \item \textit{Note:} This ordering has the advantage that, when sorted, parameters are grouped by condition
      \end{itemize}
      \item \textbf{Stem template:}\\
      $\dtype{training-set-specifications}  \ \dcode{.cpt.decpars}$
      \item \textbf{Input(s):}
      \begin{itemize}
            \item $\dtype{training-set-specification} \ \dcode{.tokdecs}$
      \end{itemize}
\end{itemize}

\subsection{Recipe: \blue\tt \%.mlr.decpars}

Multinomial logistic regression decpars.

\begin{itemize}
      \item \textbf{File format:}
      \begin{itemize}
            \item \textit{Specification:}
            \begin{itemize}
                  \item {\small $\dtype{row} \rightarrow \dtype{decision-type} \ \dcode{\ } \ \Big( \ \dtype{predictor} \ \Big( \ \dcode{,} \ \dtype{predictor} \ \Big)^* \ \Big)^? \ \dcode{\ :\ } \ \dtype{result-value}$}
                  \item {\small $\dtype{predictor} \rightarrow \dtype{predictor-name} \ \dcode{=} \ \dtype{predictor-value}$}
            \end{itemize}
            \item \textit{Example:}\\
            {\magenta\tt
            F d0\&tT=1,d0\&Top=1 : f1\&N-xX*:doctor\textunderscore 1
            }
            \item \textit{Note:} This ordering has the advantage that, when sorted, parameters are grouped by condition
      \end{itemize}
      \item \textbf{Stem template:}\\
      $\dtype{training-set-specifications}  \ \dcode{.mlr.decpars}$
      \item \textbf{Input(s):}
      \begin{itemize}
            \item $\dtype{training-set-specification} \ \dcode{.tokdecs}$
      \end{itemize}
\end{itemize}



\section{{\blue\tt \%.parweights} (formerly {\red\tt \%.model})}

Generates list of estimated parameter weights of a model. There are separate formats for ordinary conditional probability and logistic regression parameters.


\subsection{Recipe: \blue\tt \%.cpt.parweights}

Conditional probability parweights.

\begin{itemize}
      \item \textbf{File format:}
      \begin{itemize}
            \item \textit{Specification:}
            \begin{itemize}
                  \item {\small $\dtype{row} \rightarrow \dtype{decision-type} \ \ \smash{\Big( \ \dcode{\ } \ \dtype{predictor-value} \ \Big)^*} \ \dcode{ :} \ \ \smash{\Big( \ \dcode{\ } \ \dtype{result-value} \ \Big)^+} \ \dcode{ = } \ \dtype{weight}$}
            \end{itemize}
            \item \textit{Example:}\\
            {\magenta\tt
            P 2 0 1 VP : VT = 0.4237643
            }
            \item \textit{Note:} This ordering has the advantage that, when sorted, parameters are grouped by condition
      \end{itemize}
      \item \textbf{Stem template:}\\
      $\dtype{training-set-specifications}  \ \dcode{.cpt.parweights}$
      \item \textbf{Input(s):}
      \begin{itemize}
            \item $\dtype{training-set-specification} \ \dcode{.cpt.decpars}$
      \end{itemize}
\end{itemize}

\subsection{Recipe: \blue\tt \%\_mlr.parweights}

Multinomial logistic regression parweights.

\begin{itemize}
      \item \textbf{File format:}
      \begin{itemize}
            \item \textit{Specification:}
            \begin{itemize}
                  \item {\small $\dtype{row} \rightarrow \dtype{decision-type} \ \dcode{\ } \ \dtype{predictor-name} \ \dcode{\ :\ } \ \dtype{result-value} \ \dcode{\ =\ } \ \dtype{weight}$}
            \end{itemize}
            \item \textit{Example:}\\
            {\magenta\tt
            F d0\&tT : f0\&N-aD:doctor\textunderscore 1 = 0.318317
            }
            \item \textit{Note:} This ordering has the advantage that, when sorted, parameters are grouped by condition
      \end{itemize}
      \item \textbf{Stem template:}\\
      $\dtype{training-set-specifications} \ \dcode{.} \ \dtype{L2-regularization-factor} \ \dcode{\_} \ \dtype{num-iterations} \ \dcode{\_mlr.parweights}$
      \item \textbf{Input(s):}
      \begin{itemize}
            \item $\dtype{training-set-specification} \ \dcode{.decpars}$
      \end{itemize}
\end{itemize}







%%%%%%%%%%%%%%%%%%%%%%%%%%%%%%%%%%%%%%%%%%%%%%%%%%%%%%%%%%%%%%%%%%%%%%%%%%%%%%%%
%%%%%%%%%%%%%%%%%%%%%%%%%%%%%%%%%%%%%%%%%%%%%%%%%%%%%%%%%%%%%%%%%%%%%%%%%%%%%%%%
%%%%%%%%%%%%%%%%%%%%%%%%%%%%%%%%%%%%%%%%%%%%%%%%%%%%%%%%%%%%%%%%%%%%%%%%%%%%%%%%
\chapter{RESOURCE-RT}
%%%%%%%%%%%%%%%%%%%%%%%%%%%%%%%%%%%%%%%%%%%%%%%%%%%%%%%%%%%%%%%%%%%%%%%%%%%%%%%%
%%%%%%%%%%%%%%%%%%%%%%%%%%%%%%%%%%%%%%%%%%%%%%%%%%%%%%%%%%%%%%%%%%%%%%%%%%%%%%%%
%%%%%%%%%%%%%%%%%%%%%%%%%%%%%%%%%%%%%%%%%%%%%%%%%%%%%%%%%%%%%%%%%%%%%%%%%%%%%%%%

{\tt RESOURCE-RT} contains methods for aggregating measures from time-series psycholinguistic experiments into a single data table.
It also contains convenience recipes for a number of reading-time experiment types.

Note that in order to set up a new experiment using the targets in {\tt RESOURCE-RT}, you must first supply recipes to create the following input files from your experiment data:

\begin{itemize}
\item {\tt\blue $\dtype{corpus} \dcode{.lineitems}$}: parser-tokenized sentences
\item {\tt\blue $\dtype{corpus} \dcode{.linetoks}$}: experiment-tokenized sentences
\item {\tt\blue $\dtype{corpus} \dcode{.src.evmeasures}$}: space-delimited table of event data from the experiment source (IMPORTANT: this must contain {\tt\blue sentid} and {\tt\blue sentpos} columns)
\item Parameter files:
\begin{itemize}
\item {\tt\blue $\dtype{corpus} \dcode{.merge\_tables.params}$}: field(s) to use for inner merge between item and event data (typically: {\tt\blue word sentid sentpos})
\item {\tt\blue $\dtype{corpus} \dcode{.accumulateMetrics.params}$}: field(s) to accumulate over saccade regions in eye movement studies (can be empty if not relevant) 
\item {\tt\blue $\dtype{corpus} \dcode{.rm\_bad\_toks.params}$}: params to pass to {\tt\blue resource-rt/scripts/rm$\_$bad$\_$toks.py} (see script for details) 
\end{itemize}
\end{itemize}

Generally, these recipes will be supplied by creating and including a Makefile in a directory called {\tt RESOURCE-$\dtype{corpus}$}.
For example, for the Natural Stories corpus, these recipes are provided by {\tt RESOURCE-NATURALSTORIES}, which must be included before {\tt RESOURCE-RT} in the experiment directory's Makefile.

%%%%%%%%%%%%%%%%%%%%%%%%%%%%%%%%%%%%%%%%
\section{{\blue\tt \%.tokmeasures}}

Generates a data table containing space-delimited columns of measurements for each token in a corpus.

\begin{itemize}
      \item \textbf{File format:}
      \begin{itemize}
            \item \textit{Specification:}\\
                  \begin{itemize}
                  \item $\dtype{row} \rightarrow \Big( \ \dtype{token} \ \Big( \ \dcode{\ } \ \dtype{measure} \ \Big)^* \ \Big)^?$
                  \end{itemize}
            \item \textit{Example:}\\
                  {\magenta\tt
                  word totsurp ...\\
                  Are 10.2871 ...\\
                  n't 2.1029 ...
                  }
      \end{itemize}
\end{itemize}

\subsection{Recipe: {\blue\tt \%.pcfg.tokmeasures}}
Generates a table of complexity metrics from the output of an incremental parser.

\subsection{Recipe: {\blue\tt \%.t.pcfg.tokmeasures}}
Generates a table of complexity metrics from the output of an incremental parser, along with timestamps for each event in the data.
Timestamps must be provided by corpus resource named {\tt\blue $\dtype{corpus} \dcode{.time.tokmeasures}$}.
This target is a prerequisite for e.g.\ hemodynamic response convolution for fMRI studies.

\subsection{Recipe: {\blue\tt \%.syn.tokmeasures}}
Generates all token-level syntactic predictors, including parser complexity metrics as well as a number of additional predictors computed from gold trees.
Requires a corpus resource called {\tt\blue $\dtype{corpus} \dcode{.gold.linetrees}$}, and can also use coreference annotations if a {\tt\blue $\dtype{corpus} \dcode{.gold.coref.linetrees}$} file is also supplied.
If creating hand-corrected trees is not feasible, silver trees can be used by simply generating a GCG15+-style parse using any Modelblocks-supported parser and renaming the output to {\tt\blue $\dtype{corpus} \dcode{.gold.linetrees}$}.
Your {\tt RESOURCE-$\dtype{corpus}$} directory must supply targets to create or copy the gold/silver trees to the appropriate location (usually the {\tt\blue genmodel} directory of the experiment).


%%%%%%%%%%%%%%%%%%%%%%%%%%%%%%%%%%%%%%%%
\section{{\blue\tt \%.itemmeasures}}

Generates a data table containing space-delimited columns of measurements for each item in a corpus.

\begin{itemize}
      \item \textbf{File format:}
      \begin{itemize}
            \item \textit{Specification:}\\
                  \begin{itemize}
                  \item $\dtype{row} \rightarrow \Big( \ \dtype{item} \ \Big( \ \dcode{\ } \ \dtype{measure} \ \Big)^* \ \Big)^?$
                  \end{itemize}
            \item \textit{Example:}\\
                  {\magenta\tt
                  word fwprob5 ...\\
                  Aren't -11.0926 ...\\
                  }
      \end{itemize}
\end{itemize}

\subsection{Recipe: \blue\tt \%.itemmeasures}
Rolls tokens from a {\tt\blue \%.syn.tokmeasures} file into items (summing relevant columns along the way) and performs an inner merge with \textit{n}-gram surprisal metrics.

\subsection{Recipe: \blue\tt \%.ngram.itemmeasures}
Computes \textit{n}-gram surprisal columns for the items in the corpus using SRILM or KENLM.


%%%%%%%%%%%%%%%%%%%%%%%%%%%%%%%%%%%%%%%%
\section{{\blue\tt \%.evmeasures}}

Generates a data table containing space-delimited columns of measurements for each event (unit of stimulus) in a psycholinguistic experiment.

\begin{itemize}
      \item \textbf{File format:}
      \begin{itemize}
            \item \textit{Specification:}\\
            \begin{itemize}
                  \item $\dtype{row} \rightarrow \Big( \ \dtype{event} \ \Big( \ \dcode{\ } \ \dtype{measure} \ \Big)^* \ \Big)^?$
            \end{itemize}
            \item \textit{Example:}\\
            {\magenta\tt
            word subject RT ...\\
            Aren't A 187 ...\\
            Aren't B 411 ...\\
            }
      \end{itemize}
\end{itemize}

\subsection{Recipe: \blue\tt \%.full.evmeasures}
Performs an inner merge between item-level and event-level data, along with some post-processing (spillover, accumulation, removal of bad events).

\subsection{Recipe: \blue\tt \%.filt.evmeasures}
Extracts relevant columns from {\blue\tt \%.full.evmeasures}.
Designed for use as a preprocess to LME fitting, since it can dramatically reduce memory overhead, especially with multiple regressions run in parallel.
The {\blue\tt \%.rdata} target automatically infers the relevant columns based on the model specification and the command line arguments to the regression script, so this target generally does not need to be made directly.

\subsection{Recipe: \blue\tt \%.filt.evmeasures}
Generates surprisal measures from multiple grammar specifications and pastes them into a single data table.

%%%%%%%%%%%%%%%%%%%%%%%%%%%%%%%%%%%%%%%%%%%%%%%%%%%%%%%%%%%%%%%%%%%%%%%%%%%%%%%%
%%%%%%%%%%%%%%%%%%%%%%%%%%%%%%%%%%%%%%%%%%%%%%%%%%%%%%%%%%%%%%%%%%%%%%%%%%%%%%%%
%%%%%%%%%%%%%%%%%%%%%%%%%%%%%%%%%%%%%%%%%%%%%%%%%%%%%%%%%%%%%%%%%%%%%%%%%%%%%%%%
\chapter{RESOURCE-LMEFIT}
%%%%%%%%%%%%%%%%%%%%%%%%%%%%%%%%%%%%%%%%%%%%%%%%%%%%%%%%%%%%%%%%%%%%%%%%%%%%%%%%
%%%%%%%%%%%%%%%%%%%%%%%%%%%%%%%%%%%%%%%%%%%%%%%%%%%%%%%%%%%%%%%%%%%%%%%%%%%%%%%%
%%%%%%%%%%%%%%%%%%%%%%%%%%%%%%%%%%%%%%%%%%%%%%%%%%%%%%%%%%%%%%%%%%%%%%%%%%%%%%%%

{\tt RESOURCE-LMEFIT} contains methods for statistical analysis of psycholinguistic measures. The methods are primarily designed for use in reading time studies but should be generalizable to other kinds of dependent measures, as long as those measures can be associated with words/tokens in the stimuli.


%%%%%%%%%%%%%%%%%%%%%%%%%%%%%%%%%%%%%%%%
\section{{\blue\tt \%.rdata} files}

Generates saved binary of fitted linear mixed-effects regression model, along with a human-readable summary file in {\blue\tt \%.lmefit} (same stem, different extension).

\begin{itemize}
\item\textbf{File format:} \textit{Binary file}
\item\textbf{Stem template:}\\
$\dtype{evmeasures} \dcode{.} \dtype{LME-formula-name} \dcode{.} \dtype{LME-cli-args} \dcode{.rdata}$

\begin{itemize}
\item
$\dtype{evmeasures}$: Stem of a {\tt\blue *.full.evmeasures} data file from which to extract predictors.

$\dtype{LME-formula-name}$: Basename of a {\tt\blue *.lmeform} file.
For example, if the LME formula file is named {\tt\blue spr.lmeform}, its formula name will be {\tt\blue spr}.
The formula file must have the following structure:
\begin{itemize}
\item \textbf{Line 1:} Name of dependent variable (e.g.\ {\tt\blue fdur})
\item \textbf{Line 2:} List of fixed effects (e.g.\ {\tt\blue z.(sentid) + z.(sentpos) + z.(wlen)})
\item \textbf{Line 3:} List of by-subject random effects --- do not include {\tt\blue | subject)}; this will be handled by the script (e.g.\ {\tt\blue z.(sentpos) + z.(wlen)})
\item \textbf{Line 4: (optional)} Any additional random effects specifications
\end{itemize}
The formula file must live in your experiment's {\tt\blue scripts} directory.
For an example of a working formula file, see {\tt\blue resource-rt/scripts/spr.lmeform}.

$\dtype{LME-cli-args}$: List of command line arguments to pass to the regression script, with underscores used in place of spaces.
Most importantly, use {\tt\blue -a} to pass field names to add to the by-subject random effects structure only and {\tt\blue -A} to pass field names to add as both fixed and by-subject random effects.
Lists are delimited by {\tt\blue +} (e.g.\ to add two columns, {\tt\blue dlt} and {\tt\blue noF}, to the model, use {\tt\blue -A\_ dlt+noF}).
For detailed documentation of available arguments and their functions, run: {\tt\blue resource-lmefit/scripts/evmeasures2lmefit.r -h}.
\end{itemize}
\item\textbf{Inputs:}
\begin{itemize}
\item Data:\footnote{To reduce memory overhead, only the relevant columns will be extracted from this file prior to loading into the regression script} $\dtype{evmeasures} \dcode{.full.evmeasures}$
\item Model specification: $\dtype{LME-formula-name} \dcode{.lmeform}$ 
\end{itemize}
\end{itemize}


%%%%%%%%%%%%%%%%%%%%%%%%%%%%%%%%%%%%%%%%
\section{{\blue\tt \%.lmefit} files}

Contains summary data from a fitted linear mixed-effects regression model. Generated as a secondary output of {\blue\tt \%.rdata}, but for convenience can be called as a Make target in addition to {\blue \tt \%.rdata} with identical results.

\begin{itemize}
\item\textbf{File format:} \textit{Human-readable}
\end{itemize}

%%%%%%%%%%%%%%%%%%%%%%%%%%%%%%%%%%%%%%%%
\section{{\blue\tt \%.lrtsignif} files}

Generates a human-readable summary of results of likelihood ratio testing (LRT) two linear mixed-effects regression models.

\begin{itemize}
\item\textbf{File format:} \textit{Human-readable}
\end{itemize}

\subsection{Recipe: {\blue\tt \%.lrtsignif}}
Single-predictor LRT.

\begin{itemize}
\item\textbf{Stem template:}\\
$\dtype{evmeasures} \dcode{.} \dtype{LME-formula-name} \dcode{.} \dtype{main-effect} \dcode{.} \dtype{LME-cli-args} \dcode{.} \dtype{LRT-cli-args} \dcode{.lrtsignif}$
\begin{itemize}
\item
$\dtype{evmeasures}$: See {\tt\blue \%.rdata} for description

$\dtype{LME-formula-name}$: See {\tt\blue \%.rdata} for description

$\dtype{main-effect}$: Name of main effect (e.g.\ {\tt\blue dlt})

$\dtype{LME-cli-args}$: See {\tt\blue \%.rdata} for description.
Note that this target excludes args for adding effects (i.e.\ {\tt\blue -a} and {\tt\blue -A}), since these will be computed automatically by Make.

$\dtype{LRT-cli-args}$: Command line arguments to LRT, if any (if none, leave empty keeping period as delimiter)

\end{itemize}

\item\textbf{Inputs:}
\begin{itemize}
\item Baseline model: $\dtype{evmeasures} \dcode{.} \dtype{LME-formula-name} \dcode{.-a\_} \dtype{main-effect} \dcode{\_} \dtype{LME-cli-args} \dcode{.rdata}$ 
\item Main effect model: $\dtype{evmeasures} \dcode{.} \dtype{LME-formula-name} \dcode{.-A\_} \dtype{main-effect} \dcode{\_} \dtype{LME-cli-args} \dcode{.rdata}$ 
\end{itemize}
\end{itemize}

\subsection{Recipe: {\blue\tt \%.diamond.lrtsignif}}

Two-predictor LRT.
Contains results from a `diamond' LRT comparing two predictors against a baseline and against each other (i.e.\ four LRT comparisons rather than two).

\begin{itemize}
\item\textbf{Stem template:}\\
{\small $\dtype{evmeasures} \dcode{.} \dtype{LME-formula-name} \dcode{.} \dtype{effect1\_effect2} \dcode{.} \dtype{LME-cli-args} \dcode{.} \dtype{LRT-cli-args} \dcode{.diamond.lrtsignif}$}

\begin{itemize}
\item
%See {\tt\blue \%.lrtsignif} for details about stem template components.
The only difference is that {\tt\blue \%.diamond.lrtsignif} evaluates two main effects instead of one, and that these must be supplied with an underscore delimiter as $\dtype{effect1\_effect2}$.
\end{itemize}

\item\textbf{Inputs:}
\begin{itemize}
\item Baseline: $\dtype{evmeasures} \dcode{.} \dtype{LME-formula-name} \dcode{.-a\_} \dtype{effect1+effect2} \dcode{\_} \dtype{LME-cli-args} \dcode{.rdata}$
\item Effect 1: $\dtype{evmeasures} \dcode{.} \dtype{LME-formula-name} \dcode{.-a\_} \dtype{effect2} \dcode{\_-A\_} \dtype{effect1} \dcode{.} \dtype{LME-cli-args} \dcode{.rdata}$
\item Effect 2: $\dtype{evmeasures} \dcode{.} \dtype{LME-formula-name} \dcode{.-a\_} \dtype{effect1} \dcode{\_-A\_} \dtype{effect2} \dcode{.} \dtype{LME-cli-args} \dcode{.rdata}$
\item Both: $\dtype{evmeasures} \dcode{.} \dtype{LME-formula-name} \dcode{.-A\_} \dtype{effect1+effect2} \dcode{\_} \dtype{LME-cli-args} \dcode{.rdata}$
\end{itemize}
\end{itemize}


\bibliography{bibliography}

\end{document}
